\documentclass[a4paper12pt,titlepage]{article}
\usepackage{titling}
\usepackage[framed,numbered,autolinebreaks,useliterate]{mcode}
\usepackage{amsmath}
\usepackage{breqn}
\usepackage{graphicx}
\usepackage{array}
\usepackage{enumitem}
\usepackage[margin=2.0cm]{geometry}
\usepackage{booktabs}
\usepackage{amsmath}
\usepackage{amsfonts}
\usepackage{amssymb}
\title{\textbf{MTH 264 Project II}}

\author{Kemal Ficici\\Faruk Yaylagul} 

\setcounter{page}{1}
\begin{document}
\maketitle

\clearpage


\section{Problems}

Use the following methods to approximate definite integrals:
    \begin{enumerate}
        \item Simpson's Rule
        \item Composite Midpoint Rule
        \item Alternative Extended Simpson's Rule
    \end{enumerate}
    
\subsection{Integral Approximation}
Use the listed methods to approximate the following integrals.
Find the minimum \textit{N} partitions to yield 4 decimal places correctly.
\\ \textbf{Note:} Only use even values of \textit{N}, starting with \textit{N}=8.
\\ \textit{a}=0.0000000001
\begin{enumerate}[label=(\alph*)]
\item {\Large $\int_{a}^{\pi /2} \frac{x}{sin(x)} dx$}
\item {\Large$\int_{a}^{\pi /2} \frac{e^{x} - 1}{sin(x)} dx$}
\item {\Large$\int_{a}^{1} \frac{\arcsin{x}}{x} dx$}
\end{enumerate}

\subsection{Arclength}
Use the listed methods to compute the arclength of \textit{f}(x) between [\textit{a, b}].  Find the minimum \textit{N} partitions to yield 4 decimal places correctly.
\begin{enumerate}[label=(\alph*)]
\item {\Large $f(x) = \frac{x}{sin(x)}$ from $[a, \frac{\pi}{2}]$}
\item {\Large $f(x) = \frac{e^{x} - 1}{sin(x)}$ from $[a, \frac{\pi}{2}]$}
\item {\Large $f(x) = \frac{\arcsin{x}}{x}$ from $[a, 1]$}
\end{enumerate}

\subsection{Volume}
Use the listed methods to compute the volume of the solid generated by rotating \textit{f}(x)  between [\textit{a, b}] from the previous problem along the x-axis.  Find the minimum \textit{N} partitions to yield 4 decimal places correctly.

\clearpage

\section{Solutions}
    \subsection{Integral Approximation}
        Integral approximation was done using the assigned methods. The table below lists the amount of \textit{N} partitions required to reach 4 “correct” decimal places. There is no guarantee that the approximation methods converged on the correct solution, however we assume the digit to be “correct” if it hasn’t changed from the previous iteration with \textit{N}-2 partitions.
        
        \begin{table}[h!]
          \begin{center}
            \caption{Minimum \textit{N} to yield 4 "correct" decimal places}
            \label{tab:table1}
            \begin{tabular}{|l|c|c|c|}
              \toprule
              \textbf{} & \textbf{Simpson's Rule} & \textbf{Comp. Mid.} & \textbf{Alt. Ext. Simpson}\\
              \hline
              $\int_{a}^{\pi /2} \frac{x}{sin(x)} dx$ & 10 & 12 & 10\\
              \hline
              $\int_{a}^{\pi /2} \frac{e^{x} - 1}{sin(x)} dx$ & 12 & 20 & 12\\
              \hline
              $\int_{a}^{1} \frac{\arcsin{x}}{x} dx$ & 32 & 26 & 34\\
              \hline
            \end{tabular}
          \end{center}
        \end{table}
    \subsection{Arclength}
    The arclength formula is as follows:
    \begin{center}
     $\int_{a}^{b} \sqrt{1+f'(x)^{2}}  dx$
    \end{center}
    We use the integral approximation methods used in the previous problem to approximate this integral.\\ \\ 
    Supplying the double derivatives for the Composite Midpoint Rule when approximating the integral for arclength proved tedious, so the double derivative was instead approximated. \\ \\
    \begin{equation}
    \begin{split}
      f'(x) \approx \frac{f(x+h)-f(x)}{h} \\ \\
      f''(x) \approx \frac{f'(x+h)-f'(x)}{h}
     \end{split}
    \end{equation}
    The table below lists the amount of N partitions required to reach 4 “correct” decimal places. \\ \\

        \begin{table}[h!]
          \begin{center}
            \caption{Minimum \textit{N} to yield 4 "correct" decimal places}
            \label{tab:table2}
            \begin{tabular}{|l|c|c|c|}
              \toprule
              \textbf{} & \textbf{Simpson's Rule} & \textbf{Comp. Mid.} & \textbf{Alt. Ext. Simpson}\\
              \hline
              $f(x) = x/sin(x)$ & 10 & 16 & 10\\
              \hline
              $f(x) = \frac{e^x-1}{sin(x)}$ & 96 & 30 & 46\\
              \hline
              $f(x) = \frac{asin(x)}{x}$ & 552 & 290 & 556\\
              \hline
            \end{tabular}
          \end{center}
        \end{table}
    \clearpage
    \subsection{Volume}
        The volume of a solid generated by rotating the area under a curve about the x-axis can be determined using the following method:
        \begin{center}
            $V = \pi \int_{a}^{b} R(x)^2 dx$
            \\
            Where $R(x)$ is the function defining the distance between the function and the axis of rotation.
        \end{center}
        \\ The double derivative approximation method used in the previous problem was also used. \\[5pt]
        Using the integral approximation methods used in the previous problems, we can approximate the volume using the formulas. The table below lists the amount of N partitions required to reach 4 “correct” decimal places. 
        \begin{table}[h!]
          \begin{center}
            \caption{Minimum \textit{N} to yield 4 "correct" decimal places}
            \label{tab:table3}
            \begin{tabular}{|l|c|c|c|}
              \toprule
              \textbf{} & \textbf{Simpson's Rule} & \textbf{Comp. Mid.} & \textbf{Alt. Ext. Simpson}\\
              \hline
              $R(x) = x/sin(x)$ & 14 & 30 & 24\\
              \hline
              $R(x) = \frac{e^x-1}{sin(x)}$ & 26 & 70 & 24\\
              \hline
              $R(x) = \frac{asin(x)}{x}$ & 68 & 56 & 78\\
              \hline
            \end{tabular}
          \end{center}
        \end{table}
\clearpage
\section{Code}
    \subsection{Problem 1: Integral Approximation}
        \subsubsection{Simpson's Rule}
        \lstinputlisting[firstline=0]{problem1/simpson.m}
    \clearpage
        \subsubsection{Composite Midpoint Rule}
        \lstinputlisting[firstline=0]{problem1/compositemid.m}
     \clearpage
       \subsubsection{Alternative Extended Simpson's Rule}
        \lstinputlisting[firstline=0]{problem1/exsimp.m}
    \clearpage
    \subsection{Problem 2: Arclength}
        \subsubsection{Simpson's Rule}
        \lstinputlisting[firstline=0]{problem2/simpson.m}
    \clearpage
        \subsubsection{Composite Midpoint Rule}
        \lstinputlisting[firstline=0]{problem2/compositemid.m}
     \clearpage
       \subsubsection{Alternative Extended Simpson's Rule}
        \lstinputlisting[firstline=0]{problem2/exsimp.m}
    \clearpage
    \subsection{Problem 3: Volume}
        \subsubsection{Simpson's Rule}
        \lstinputlisting[firstline=0]{problem3/simpson.m}
     \clearpage
       \subsubsection{Composite Midpoint Rule}
        \lstinputlisting[firstline=0]{problem3/compositemid.m}
    \clearpage
        \subsubsection{Alternative Extended Simpson's Rule}
        \lstinputlisting[firstline=0]{problem3/exsimp.m}
\end{document}
